% Options for packages loaded elsewhere
\PassOptionsToPackage{unicode}{hyperref}
\PassOptionsToPackage{hyphens}{url}
%
\documentclass[
]{article}
\usepackage{amsmath,amssymb}
\usepackage{iftex}
\ifPDFTeX
  \usepackage[T1]{fontenc}
  \usepackage[utf8]{inputenc}
  \usepackage{textcomp} % provide euro and other symbols
\else % if luatex or xetex
  \usepackage{unicode-math} % this also loads fontspec
  \defaultfontfeatures{Scale=MatchLowercase}
  \defaultfontfeatures[\rmfamily]{Ligatures=TeX,Scale=1}
\fi
\usepackage{lmodern}
\ifPDFTeX\else
  % xetex/luatex font selection
\fi
% Use upquote if available, for straight quotes in verbatim environments
\IfFileExists{upquote.sty}{\usepackage{upquote}}{}
\IfFileExists{microtype.sty}{% use microtype if available
  \usepackage[]{microtype}
  \UseMicrotypeSet[protrusion]{basicmath} % disable protrusion for tt fonts
}{}
\makeatletter
\@ifundefined{KOMAClassName}{% if non-KOMA class
  \IfFileExists{parskip.sty}{%
    \usepackage{parskip}
  }{% else
    \setlength{\parindent}{0pt}
    \setlength{\parskip}{6pt plus 2pt minus 1pt}}
}{% if KOMA class
  \KOMAoptions{parskip=half}}
\makeatother
\usepackage{xcolor}
\usepackage[margin=1in]{geometry}
\usepackage{graphicx}
\makeatletter
\def\maxwidth{\ifdim\Gin@nat@width>\linewidth\linewidth\else\Gin@nat@width\fi}
\def\maxheight{\ifdim\Gin@nat@height>\textheight\textheight\else\Gin@nat@height\fi}
\makeatother
% Scale images if necessary, so that they will not overflow the page
% margins by default, and it is still possible to overwrite the defaults
% using explicit options in \includegraphics[width, height, ...]{}
\setkeys{Gin}{width=\maxwidth,height=\maxheight,keepaspectratio}
% Set default figure placement to htbp
\makeatletter
\def\fps@figure{htbp}
\makeatother
\setlength{\emergencystretch}{3em} % prevent overfull lines
\providecommand{\tightlist}{%
  \setlength{\itemsep}{0pt}\setlength{\parskip}{0pt}}
\setcounter{secnumdepth}{-\maxdimen} % remove section numbering
\newlength{\cslhangindent}
\setlength{\cslhangindent}{1.5em}
\newlength{\csllabelwidth}
\setlength{\csllabelwidth}{3em}
\newlength{\cslentryspacingunit} % times entry-spacing
\setlength{\cslentryspacingunit}{\parskip}
\newenvironment{CSLReferences}[2] % #1 hanging-ident, #2 entry spacing
 {% don't indent paragraphs
  \setlength{\parindent}{0pt}
  % turn on hanging indent if param 1 is 1
  \ifodd #1
  \let\oldpar\par
  \def\par{\hangindent=\cslhangindent\oldpar}
  \fi
  % set entry spacing
  \setlength{\parskip}{#2\cslentryspacingunit}
 }%
 {}
\usepackage{calc}
\newcommand{\CSLBlock}[1]{#1\hfill\break}
\newcommand{\CSLLeftMargin}[1]{\parbox[t]{\csllabelwidth}{#1}}
\newcommand{\CSLRightInline}[1]{\parbox[t]{\linewidth - \csllabelwidth}{#1}\break}
\newcommand{\CSLIndent}[1]{\hspace{\cslhangindent}#1}
\ifLuaTeX
  \usepackage{selnolig}  % disable illegal ligatures
\fi
\IfFileExists{bookmark.sty}{\usepackage{bookmark}}{\usepackage{hyperref}}
\IfFileExists{xurl.sty}{\usepackage{xurl}}{} % add URL line breaks if available
\urlstyle{same}
\hypersetup{
  pdftitle={Suitability prevalence area index in late quaternary explains genetic diversity in Tassel eared Squirrels},
  pdfauthor={Norma Hernandez, Angel Robles Nathan Upham},
  hidelinks,
  pdfcreator={LaTeX via pandoc}}

\title{Suitability prevalence area index in late quaternary explains
genetic diversity in Tassel eared Squirrels}
\author{Norma Hernandez, Angel Robles Nathan Upham}
\date{2023-06-27}

\begin{document}
\maketitle

\hypertarget{abstract}{%
\subsection{Abstract}\label{abstract}}

The current distributions of species do not always correspond to their
historical distributions over evolutionarily significant periods. This
is because environmental conditions are not static over time; species
tend to distribute where conditions are most favorable.

It follows from this relationship between environment and species
distribution that population size also varies with time. This variation
in population size is related to the effective population size \(N_e\),
so that an index reflecting changes in environmental conditions in
geography might be expected to be related to \(N_e\) and thus to
indicators of population structure such as the fixation index
\(F_{st}\). Thus, it is possible to relate patterns of changes in the
distribution of species to the genetic structure of their populations
using a statistical model that explains this relationship.

With this approach we can predict the geographic pattern of population
structure from environmental information. This approach is strongly
driven by advances in currently available climate simulations (Leonardi
et al. 2023; Krapp et al. 2021), as well as next generation sequencing
data, and supported by both ecological niche (Thorup et al. 2021;
Nogués-Bravo 2009) and population genetics theories (Lira-Noriega and
Manthey 2014).

In this work we propose a method to find the Suitability Prevalence Area
(SPA) as an index with a double purpose: 1) to find endemic areas to
delimit the historical distribution of the species and 2) to explain the
patterns of genetic diversity.

To obtain the SPA, we performed a historical reconstruction of the
geographical range back to 120 000 BC at 2 000 year intervals and
recorded the environmental suitability at each site.

Subsequently, we delimited historical endemic areas to locations where
the prevalence of suitability remained at 90\% during this period. From
the fixation index calculated with respect to populations in the
historical endemic areas, a statistical model of the fixation index as a
function of SPA was performed. With this statistical model, the fixation
index values were projected to the current distribution to obtain a map
with the geographic pattern of this index.

As a case study we consider squirrels (\emph{Sciurus aberti}), a species
currently distributed in disjunct patches from the southern Rocky
Mountains in the United States to the northern Sierra Madre Occidental
in Mexico for which it is possible to find reliable information on both
the genetic structure of its populations and its current distribution
(Bono et al. 2018; Burgin et al. 2018).

Our results reveal that suitability prevalence corresponds to the
fixation index of \emph{S. aberti} populations with respect to a source
population. Populations closer to the historical endemic area present a
higher genetic diversity and a lower \(F_{st}\) value. Finally, this
study allows us to add a biogeographic explanation to the results
obtained with population genetic methods and to generate maps of this
structure as tools to support conservation with a perspective that
integrates both population genetics and historical patterns of species
distribution.

\hypertarget{refs}{}
\begin{CSLReferences}{1}{0}
\leavevmode\vadjust pre{\hypertarget{ref-bono2018genome}{}}%
Bono, Jeremy M, Helen K Pigage, Peter J Wettstein, Stephanie A Prosser,
and Jon C Pigage. 2018. {``Genome-Wide Markers Reveal a Complex
Evolutionary History Involving Divergence and Introgression in the
Abert's Squirrel (Sciurus Aberti) Species Group.''} \emph{BMC
Evolutionary Biology} 18 (1): 1--17.

\leavevmode\vadjust pre{\hypertarget{ref-burgin2018many}{}}%
Burgin, Connor J, Jocelyn P Colella, Philip L Kahn, and Nathan S Upham.
2018. {``How Many Species of Mammals Are There?''} \emph{Journal of
Mammalogy}. Oxford University Press US.

\leavevmode\vadjust pre{\hypertarget{ref-krapp2021statistics}{}}%
Krapp, Mario, Robert M Beyer, Stephen L Edmundson, Paul J Valdes, and
Andrea Manica. 2021. {``A Statistics-Based Reconstruction of
High-Resolution Global Terrestrial Climate for the Last 800,000
Years.''} \emph{Scientific Data} 8 (1): 228.

\leavevmode\vadjust pre{\hypertarget{ref-leonardi2023pastclim}{}}%
Leonardi, Michela, Emily Y Hallett, Robert Beyer, Mario Krapp, and
Andrea Manica. 2023. {``Pastclim 1.2: An r Package to Easily Access and
Use Paleoclimatic Reconstructions.''} \emph{Ecography}, e06481.

\leavevmode\vadjust pre{\hypertarget{ref-lira2014relationship}{}}%
Lira-Noriega, Andrés, and Joseph D Manthey. 2014. {``Relationship of
Genetic Diversity and Niche Centrality: A Survey and Analysis.''}
\emph{Evolution} 68 (4): 1082--93.

\leavevmode\vadjust pre{\hypertarget{ref-nogues2009predicting}{}}%
Nogués-Bravo, David. 2009. {``Predicting the Past Distribution of
Species Climatic Niches.''} \emph{Global Ecology and Biogeography} 18
(5): 521--31.

\leavevmode\vadjust pre{\hypertarget{ref-thorup2021response}{}}%
Thorup, Kasper, Lykke Pedersen, Rute R Da Fonseca, Babak Naimi, David
Nogués-Bravo, Mario Krapp, Andrea Manica, et al. 2021. {``Response of an
Afro-Palearctic Bird Migrant to Glaciation Cycles.''} \emph{Proceedings
of the National Academy of Sciences} 118 (52): e2023836118.

\end{CSLReferences}

\end{document}
